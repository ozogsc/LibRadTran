\subsection{SSRadar - Single Scattering Radar \index{SSRadar}}

A simple single scattering radar simulator which assumes pure Rayleigh scattering
in a 1D atmosphere at water or ice spheres without molecular absorption. It solves 
the equation for the radar reflectivity factor $Z$
$$
Z = \sum_i N_i D_i^6
$$

directly from the given droplet distribution defined by discrete sampling intervalls
$\left(N_i, N_i + \Delta N\right)$ containing the droplet diameters $\left(D_i, D_i 
+ \Delta D\right)$ (\citet{Rinehart2010}).

Usage (outputfile is optional, output is printed to stdout by default):
\begin{Verbatim}[fontsize=\footnotesize]
ssradar inputfile [outputfile]
\end{Verbatim}
The simulation is defined in the input file with the following structure:
\subsubsection {First line:}
\begin{description}
\item[wavelength] The wavelength in mm.
\item[Z-Position] The vertical position of the radar in m.
\item[Zenith Angle] The zenith angle in degree, must not be $90$.
\item[Ground Altitude] Vertical position of the ground in m.
\end{description}
\subsubsection{Second line:}
\begin{description}
\item[Range Gates] Number of range gates.
\item[Range Gate Length] Length of a range gate in m.
\item[First Range Gate] Distance of the first range gate from the radar in m.
\item[Output] Ouput format. -1 is text only, 1 is mmclx only, 0 is both.
\end{description}
Notes: The simulator takes a straight line from the radar into the direction defined
by the zenith angle. This line is partitioned into a number of intervals (number of 
range gates) with a fixed length (range gate length) starting at a given distance 
(first range gate) from the radar. Keep in mind that the first range gate is defined by 
its distance along that line and not by its z-position, and that the length of a range
gate is the actual geometrical length.
\subsubsection{The following lines define the reflective layers, one line per layer:}
\begin{description}
\item[Height] Z-Position of the lower border of the layer in m.
\item[Thickness] Vertical thickness of the layer in m.
\item[Effective Radius] Effective Radius of the distribution in um.
\item[Distribution] What kind of distribution (mono, gamma, log, or a filename), for
details see \code{distribution} and \code{size\_distribution\_file} in chapter \ref{sec:mie}.
\item[Distribution Parameter] Distribution Parameter for gamma and log, for
details see \code{distribution} in chapter \ref{sec:mie}.
\item[Maximum Radius Factor] Upper radius cutoff factor for gamma and log, for
details see \code{n\_r\_max} in chapter \ref{sec:mie}.
\item[Temperature] Temperature of the layer in degree Celsius.
\item[Phase] Phase of the layer, water or ice.
\item[Liquid Water Content] Liquid Water Content of the layer in $g/m^3$
\end{description}
Notes: You can simulate supercooled liquid water down to $-34^{\circ}C$.
Reflective layers must not overlap or be below ground but can otherwise be 
configured freely. 

\subsubsection{Example:}
\begin{Verbatim}[fontsize=\scriptsize,frame=single]
# wvl - z-pos -   zenith angle - ground altitude
# mm  - m     -   degree       - m
  8.5   500.0       45.0        0.0
# Range Gates - RG Length - First RG - Output
# (int)         m           m          (-1,0,1)
  15            500.0       0.0        -1
# height - thickness - reff - dist - distparam - max r fact - T - phase - LWC
# m        m           um     (str)  (double)    (double)  deg C  (str)   g/m^3
  1000.0   1000.0      18.0   mono       0          0        10.  water   0.30
  2000.0    500.0      15.0   log        1.1        5         0.  water   0.10
  3000.0    500.0      12.0   gamma      7          5       -10.  water   0.05
  3500.0   2000.0      10.0   dist.dat   0          0       -20.  ice     0.02
\end{Verbatim}

